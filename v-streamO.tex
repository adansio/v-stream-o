\documentclass[12pt]{article}
\usepackage[T1]{fontenc}
\usepackage[utf8]{inputenc}
\usepackage[spanish]{babel}
\usepackage{graphicx}

\title{V-stream'o}
\author{Mutimedios - TEL-332\\\\Luis González - José Jorquera - Adan Morales}
\date{\today}

\begin{document}
\maketitle
\newpage
\tableofcontents
\thispagestyle{empty}

\newpage
\section{Introducción}
\subsection{Trasfondo}
Existen momentos durante el periodo universitario que por uno u otro motivo no es posible asistir a clases 
``claves'' para la comprensión a cabalidad de los contenidos de una asignatura, ya sea por ejemplo, un 
compromiso de fuerza mayor, o alguna eventualidad como sucedió el a\~no 2011 con el movimiento estudiantil, 
donde en varios casos no fue posible entregar completamente o explicar como corresponde alguna materia 
en particular, o quizas algún contenido es frecuentemente consultado, y el profesor desearía tener una 
explicación mas a fondo de algun tema y compartirla con sus alumnos pero bajo ciertas condiciones de uso
y de reservación de derechos.\\

Es por esto que se plantea un sistema en el cual mediante videos (grabaciones o streming media) el profesor
podrá compartir ``clases'' a sus alumnos, tal como existen ejemplos de profesores de Massachusetts Institute 
of Technology (MIT) que publican sus clases en sitios como www.youtube.com ayudando a comprensión de temas
no solo a usuarios del MIT si no que a nivel mundial.

\subsection{Resumen}
El proyecto consta de un sistema de acceso web, en el cual un profesor (registrado en el sistema) puede 
publicar videos de ciertas asignaturas o algún contenido en específico a modo de apoyo para la comprensión
de un tema en particular, y además realizar video conferencias programadas para sus alumnos. Junto con esto
el alumno podrá acceder a estos contenidos mediante un registro previo en el sistema,
este será de un ambiente similar a lo que es el sitio www.aula.usm.cl, agregando esta nueva
funcionalidad de video, junto con esto cualquier nueva duda proveniente del video, se dispondrá del mail 
del profesor para realizarle consultas, y en el caso de que el recurso sea streaming, se contará con un campo
especial de mensajeria para relizarle consultas directamente durante la clase en vivo a traves del sistema
interno V-stream'O.


\newpage
\section{Descripción General}

\subsection{Objetivos}
\subsubsection{Generales}
Como se ha mencionado anteriormente el objetivo general del sistema es acercar el conocimiento de los 
profesores a sistemas actuales de comunicación para sus alumnos, como es la publicación de video o de streming
mediante el sistema interno V-stream'O.
\subsubsection{Específicos}
\begin{enumerate}
\item Crear un portal web en el cual se puedan publicar videos asociados a asignaturas específicas o 
complementarias de la UTFSM.
\item Alumnos registrados en el sistema puedan tener acceso a los videos especificados según las
restricciones pertinentes.
\item Establecer sistema de video conferencias en la cual el profesor pueda mediante una webcam local
realizar una ``clase on-line''.
\item Los alumnos puedan observar las ``clases on-line'' y realizar consultas al profesor a través de un 
sistema de chat interno.
\end{enumerate}

\subsubsection{Problemática que enfrenta}
El Problema a enfrentar en este proyecto es el de poder ayudar en la formación académica de un alumno de la 
UTFSM, debido a veces en que por alguna razón externa a la vida académica, o quizas para los
deportistas, que deben ir a representar a la universidad en campeonatos, o dentro de los distintos talleres 
que la universidad dispone, algunos alumnos no podrán
asistir a alguna clase de real importancia dejándolo posiblemente sin algún conocimiento elemental, aun
cuando es responsabilidad de este consultar sobre los temas no comprendidos.

Tambien se puede dar el caso que el profesor no alcance a cubrir todo un tema con la profundidad que desea
dentro del tiempo estipulado pedagógicamente, por lo cual quedan ciertos temas como ``en el aire'', o
simplemente no se cubren.


\subsection{Descripción de la solución}
A modo de solución a la problematica anterior, se establece este sistema, de forma que (ayudados de la buena
voluntad de los profesores en cuanto a la contribución de material ya sea fundamental o complementario), 
eventualmente el profesor publique material
audiovisual como por ejemplo videos, ya sean de algún otro sitio como ``youtube'' o ``vimeo'' entre otros, o 
que él mismo planifique una clase en donde esta sea filmada y luego publicada en el sistema
a implementar, y así dicho alumno podrá acceder al sitio web del sistema y revisar dicho contenido, 
quizas con mayor detenimiento; o mejor aún, el profesor ayudado mediante el sistema V-stream'O podrá
acordar y luego realizar ``una clase o ayudantia on-line'', en la cual sus alumnos podrán tener acceso a 
a este, y ademaś existirá un chat en donde el alumno podrá realizar consultas que serán
revisadas por el profesor mientras se realiza el streaming.

\subsection{Tecnologías usadas}

- Servidor web Apache en sistema operativo Debian (6.0 Squeeze).\\

- Servidor de streaming en sistema operativo Ubuntu 10.04 (Lucid Lynx).\\

- Lenguaje en servidor PHP.\\

- Modelamiento de base de datos sqlDesigner.\\

- Manejo de base de datos con MySQL.\\

- Framework de desarrollo Cakephp.\\ 


\newpage
\section{Especificacón de los requerimientos}
\subsection{Usuarios del sistema}
\paragraph{Alumno\\}

Los usuarios mayoritariamente beneficiados con este sistema son los alumnos, ya que principalmente
el proyecto está enfocado a ellos, en el cual podrán obtener los recursos multimedia ya sea video
o una clase via streaming, además de poder realizar consultas mientras se desarrolla dicha clase. Las 
especificaciones son las siguientes:\\
- Computador con acceso a internet.\\
- Browser de preferencia Google-Chrome.\\
- En caso de utilizar Firefox, este debe tener soporte para
	HTML5 para poder reproducir videos, junto con los plugins adecuados, ya sea ffmpeg o biblioteca 
	flashplayer.\\

\paragraph{Profesor\\}

Los usuarios que aportarán en que el sistema cumpla el objetivo general, son lo profesores, 
encargados de publicar periódicamente videos de interés, ya sean de orden importante o complementarios, 
ademas de realizar ``clases on-line'' cuando ellos estimen conveniente, o corresponda por alguna situción
en particular; la lista de especificaciones se detalla a continuación:\\
- Computador con acceso a internet.\\
- Browser de preferencia Google-Chrome.\\
- En caso de utilizar Firefox, este debe tener soporte para
	HTML5 para poder reproducir videos, junto con los plugins adecuados, ya sea ffmpeg o biblioteca 
	flashplayer.\\
- Webcam y micrófono para poder realizar el streaming.\\
- Obtener el archivo de configuracón para poder comenzar el stramening.\\

\subsection{Descripción de los requerimientos}
\subsubsection{Funcionales}

- Permitir al usuario no registrado acceder a cierto contenido público.\\

- Permitir al usuario Alumno registrarse en el sistema, con un username del tipo nombre.alumno@alumnos.usm.cl,
	Password con un mínimo de 6 caracteres, Nombre, Rut, Rol, código de Carrera. Dependerá del alumno realizar
	una correcta inscripción de su cuenta. en caso de quere modificar algún campo luego de la inscripción,
	deberá contactarse con un administrador.\\

- Permitir al usuario Alumno ver a los profesores disponibles y sus publicaciones.\\

- Permitir al usuario Alumno ver los videos publicados por el profesor.\\

- Permitir al usuario Alumno mantener un historial con las fechas de los útimos videos observados.\\ 

- La creación de un usuario tipo Profesor debe ser a través de un Administrador.\\

- Permitir al usuario Profesor modificar su contenido.\\

- Permitir al usuario Profesor publicar contenido de tipo ``POST''.\\

- Permitir al usuario Profesor modificar o editar los contenidos de tipo ``POST''.\\

- Permitir al usuario Profesor cargar videos en el servidor para una eventual reproducción por parte los
	alumnos.\\

- Permitir al usuario Profesor eliminar videos subidos previamente.\\

- Permitir al usuario Profesor realizar streaming de video y audio.\\

- Permitir al usuario Profesor observar un chat mientras se realiza el streaming, para eventualmente 
	poder responder posibles interrogantes de parte de los alumnos.\\

\newpage
\subsubsection{No Funcionales}

\paragraph{- De Buen rendimiento} en cuanto a la calidad del servicio, ya sea su disponibilidad para acceder a los 
contenidos o como para poder participar de una video conferencia. Si bien no se ha comprobado a gran escala
el correcto funcionamiento, este ha funcionado con 4 usuarios concurrentemente conectados.\\

\paragraph{- Seguro} en el sentido de integridad de los datos y su consistencia, para poder acceder a los contenidos
pertinentes según los permisos que se tengan. Por medio del model MVC, se han separado correctamente los datos, 
además de un correcto planteamiento de la base de datos en la cual no existen datos repetidos, en los cuales
pudiesen haber incongruencias injsutificadas.\\

\paragraph{- Accesible} y de rápida asimilación, que sea una interfaz intuitiva para el usuario, sin mayores
ambigüedades. En si el dise\~no es bastante sencillo, de rápida navegabilidad.\\

\paragraph{- Escalable}, si bien es un modelo a una peque\~na escala, el dise\~no debe soportar una amplia 
expansión en cuanto a gran cantidad de usuarios utilizando el servicio.\\

\paragraph{- Administrable} por parte, tanto del profesor que será quien dispondrá del contenido para su posterior
reproducción. En esta segunda fase se han publicado ciertos campos, quizas para una eventual mejora del sistema
se podrian a\~nadir mas campos y otras formas de mostrar la información que se requiera.\\

\newpage
\subsection{Tareas de Usuario}
\paragraph{Alumno\\}

El Alumno podrá hacer las peticiones a su respectivo profesor para que haga aportes tanto de temas 
centrales como de temas complementarios, y una vez que se disponga del material, este podrá consumir
de este recurso para su beneficio propio. Tambien durante el transcurso de una ``clase on-line'' podrá,
mediante un tipo de chat, realizar consultas, las cuales, evuentualmente serían respondidas por el profesor en ese momento.\\

\paragraph{Profesor\\\\}

La principal tarea del profesor será la de ir manteniendo activo ``su portal'' e ir haciendo aportes
tanto a sus alumnos, como del ramo en general, esto es, ir subiendo contenido multimedia al servidor de 
v-stream'o, y para ciertas ocaciones programar clases ``on-line'' y realizar el streaming correspondiente
mediante este sistema.\\

\subsection{Funciones del Sistema}

\subsubsection{En el lado del servidor}

-	Mantención de cuentas del cliente en general, ya sea para alumno, profesor, por parte de un administrador.\\

-	Gestor para la creación de cuentas para alumno.\\

- 	Atención para la gestión en la creacion de cuenta para profesor.\\

-	Gestión de los accesos y permisos correspondientes para cada usuario en particular.\\

-	Control de las sesiones de cada cliente conectado, de forma tal de brindarle el servicio correspondiente.\\

-	Almacenamiento de los recursos multimedia almacenados por el cliente (profesor) para una posterior
reproducción de otro cliente (alumno u otro profesor).\\

-	Gestionar las clases on-line mediante el streaming.\\

-	En el caso del streaming, que brinde buen servicio a los clientes conectados, tanto a los usuarios
conectados al streaming, como a los que no lo estan.\\

-	Almacenamiento de las publicaciones por parte de los profesores, ya sea para realizar algún comentario,
	o para una programar una futura video-conferencia.\\

-	Creación, edición y eliminación de un elemento de tipo publicación.\\

- 	Creacion y eliminación de un elemento de tipo video.\\


\subsubsection{En el lado del cliente}

-	Al momento de ingresar al sistema autenticándose, corroborar que el usuario está ya registrado en
el sistema para poder acceder a los contenidos respectivos.\\

- 	La creación de una cuenta de forma única en cuanto a un identificador de usuario, junto con el username,
	(a modo de correo electrónico), prevenir la creación de una cuenta con los mismos usuarios de correo.\\

-	En caso que el usuario no se encuentre en lo base de datos, podrá ingresar a ellos mediante un 
registro, para el caso de los alumnos, será responsabilidad de ellos la correcta creación de su cuenta,
en caso de algún desperfecto, este usuario deberá contactar al administrador del sistema para realizar 
modificaciones, para el caso de un profesor, este se econtrará a cargo de un administrador quien irá validando 
los datos entregados.\\

-	Una vez que el cliente está autenticado, dependiendo del tipo de usuario (alumno, profesor o 
administrador), podrá realizar diferentes actividades, y distintos permisos, estos son detallados
más adelante de mejor forma en el esquema navegacional.\\

-	Para el caso del cliente-profesor, podrá ver y publicar videos, publicar posts, realizar la programación 
de las futuras clases on-line y eventualmente llevarlas a cabo mediante el streaming.\\

-	Para el caso del cliente-alumno, podrá ver videos publicados según un profesor,
podrá ver clases on-line y tambien realizar consultas al profesor respectivo durante el 
streaming, mediante un chat interno.

\newpage
\section{Dise\~no de la interfaz de usuario}
\subsection{Esquema Navegacional}
Diagrama de navegabilidad: muestra la navegabilidad de parte de un usuario cliente del servicio.\\
Esquema Navegacional en la Figura ~\ref{fig:Nav}.

\subsection{Prototipos de Pantallas}
Vista de Administrador en Figura~\ref{fig:Admin}

Vista de Loginen Figura~\ref{fig:login}

Vista de lista de profesores en Figura~\ref{fig:lista_prof_admin}

Vista de lista de publicaciones en Figura~\ref{fig:lista_pub}

Vista de lista de videos en Figura~\ref{fig:lista_vid}

Vista de agregar posts en Figura~\ref{fig:add_post}

Vista de reproducir video en Figura~\ref{fig:ver_post}

Vista de Streaming en Figura~\ref{fig:streaming}

Vista de reproducir video en Figura~\ref{fig:ver_video}

Vista de reproducir video en Figura~\ref{fig:add_video}



\newpage
\section{Dise\~no del Sistema}
\subsection{Dise\~no de la Arquitectura}
El dise\~no de la arquitectura implementado es el modelo de tres capas, Model-View-Controller (MVC), debido a la abstracción que se realiza entre ellas, operando en cada capa por separado.\\
Estas capas constan de:\\
-	Capa de Presentación.\\
-	Capa de Negocio.\\
-	Capa de Modelo o de base de datos.\\

Al realizar un enfoque a como se ha implementado en el sistema, junto con el framework se ha implementado de mejor 
forma este modelo, debido a como está estructurado, y como es la distribucion dentro del proyecto de los archivos.

En el caso del modelo usuario se ocupa el User.php donde se especifican las realciones a la base de datos y la 
validacion de datos, en su controlador están las funciones que maneja los datos. En el directorio View/User se 
encuentran las interfaces de usuario para el modelo.\\

\subsection{Dise\~no Lógico}
\subsubsection{Diagrama de Clases}
Diagrama de Clases: contiene las clases relacionadas en la aplicación, se muestra que un usuario generaliza al 
administrador, profesor y alumno, donde pueden loguearse, insertar un nuevo integrante y cerrar sesión. 
Un usuario tiene muchos videos relacionados, donde se pueden subir, ver y borrar videos. Para el streamig se 
ocupa las publicaciones, donde se pude publicar, borrar y editar los post, también se pude transmitir el 
streaming. El lenguaje usado para modelar es UML. \\

Diagrama de Clases en la figura ~\ref{fig:Clases}.

\subsubsection{Diagramas de Secuencias}
Diagrama de Secuencia: muestra la interacción entre los actores y el sistema, en los distintos casos de uso
del software.\\

Diagrama ver publicacion en las Figura~\ref{fig:al_publ}.

Diagrama ver video en la Figura~\ref{fig:al_vid}.

Diagrama eliminar usuario en la Figura~\ref{fig:sec_del_user}.

Diagrama borrar video en la Figura~\ref{fig:del_vid}.

Diagrama ingresar usuario en la Figura~\ref{fig:ing_user}.

Diagrama publicacion profesor en la Figura~\ref{fig:sec_pub_prof}.

Diagrama subir video en la Figura~\ref{fig:sub_vid}.

Diagrama streaming en profesor en la Figura~\ref{fig:str_pr}.


\subsubsection{Modelo de Datos}
Modelo de Datos: muestra las tablas alojadas en la base de datos, donde se muestra la relación entre las 
tablas por medio del lenguaje Entidad-Relación. \\

Modelo de Datos en la Figura ~\ref{fig:moddatos}.



\newpage
\section{Implementación}
\subsection{Descripción de las Componentes}
\paragraph{Implementación\\}
Códigos de Modelos, en estos códigos se representan la estructura, asociaciones y validaciones de datos de los 
Modelos, que representan las tablas en la Base de Datos, a continuación se describen los más importantes.\\
User.php: Contiene la relación ``hasmany''  con el Modelo Video, Historiale y Post;  un usuario tiene muchos videos.
La relación ``belongto'' con el Modelo carrera, un Usuario pertenece a una carrera. Luego se describen la validación 
de los datos más importantes de entrada, el username que debe existir y debe ser un mail, la password que debe 
existir y con un mínimo de seis caracteres, donde se guarda encriptada en la Base de datos.\\

\footnotesize
\begin{verbatim}
#app/Model/User.php
<?php
class User extends AppModel {
    public $name = 'User';
	var $hasMany = array('Video' => array('className' => 'Video', 
                    		'foreignKey'=> 'user_id',
                      		'order'    => 'Video.id DESC',            
	   		         		'dependent'=> true        ),
		 	      		'Post' => Array('ClassName' => 'Post',
							'foreignKey' => 'user_id'),
							'Historiale' => array('className' => 'Historiale', 
												'foreignKey' => 'user_id',
							'order' => 'Historiale.id DESC',            
						    'limit' => '5',           
	                        'dependent'=> true));
	var $belongsTo = array('Carrera' => array(  'className'=> 'Carrera', 
												'foreignKey'=>'carrera_id' )); 
					 public $displayField = 'username';	
	var $validate = array( 'username' => array('email' => array('rule' => 'email',
	                       'required' => true, 
	                       'message' => 'Introduzca su nombre de usuario.'),
						   'isUnique'=>array('rule'=>'isUnique',
                           'required' => true, 
			               'message' => 'El nombre ya ha sido ocupado'), ),
						   'password' => array('rule' => array('minLength', '6' ),
	                       'required' => true,
                           'message' => 'Debe contener minimo 6 caracteres')
                           );

	public function beforeSave($options = array()) {
		if (isset($this->data[$this->alias]['password'])) {
			$this->data[$this->alias]['password'] = 
						AuthComponent::password($this->data[$this->alias]['password']);
		}
	    return true;
	}

}
\end{verbatim}

Video.php: Contiene la relación belongto con el Modelo User, un videoo pertenece a un usuario.\\
\footnotesize
\begin{verbatim}
<?php
class Video extends AppModel {
	public $name = 'Video';
    var $uses = array('User','Historiale');
	var $belongsTo = array('User' => array('className' => 'User', 
						   'foreignKey'   => 'user_id' ));  
	var $validate = array('nombre' => array('rule' => 'notEmpty'), 
							'user_id' => array('rule' => 'notEmpty'));
}
?>
\end{verbatim}

Post.php: Contiene la relación belongto con el Modelo User, un post pertenece a un usuario.\\
\footnotesize
\begin{verbatim}
<?php
class Post extends AppModel {
	public $name = 'Post';
	public $validate = array('title' => array(
	            			'rule' => 'notEmpty'),
					        'body' => array(
			            	'rule' => 'notEmpty');
	var $belongsTo = array('User' => array(  'className'=>'User', 
											'foreignKey'   => 'user_id',
											'order' => 'Post.id DESC'));
}?>
\end{verbatim}

Historiale.php: Contiene la relación belongto con el Modelo User, un historial pertenece a un usuario.\\

\footnotesize\begin{verbatim}
<?php
class Historiale extends AppModel {
	public $name = 'Historiale';
    var $uses = array('User','Video');
    var $belongsTo = array('User' => array('className' => 'User',
						'foreignKey'   => 'user_id' ));
   }	
?>
\end{verbatim}

Código de Controladores, en estos códigos es donde se maneja la capa de negocios de la aplicación, se trabaja 
con las consultas a la base de datos y se procesan las variables involucradas, en funciones definidas, 
a las cuales se le asocia una vista, inicializan con el index.\\

Admin Controller.php: es este código se contiene las funciones para realizar las vistas de videos y profesores, 
y la capacidad para borrarlos.\\
\footnotesize
\begin{verbatim}
<?php
class AdminController extends AppController {
	var $helpers = array('Form', 'Html', 'Time');
	var $uses = array('Alumno', 'Profe', 'Video', 'User','Carrera');
	public function isAuthorized($user){
		if(in_array($this->action, array('index','view','viewuser', 
					'deleteuser', 'deletevideo', 'lista_profe', 
					'lista_videos' )))
		{
			if($user['tipo'] == 2 ){
				return true;
			}
		}
		return false;
	}
	
	function index() {

	}

	function deletevideo($id) {
		if ($this->Video->delete($id)){
			$this->Session->setFlash('El video with id: '.$id.' fue borrado.');
			$this->redirect(array('action'=>'index'));
		}
	}

	function deleteuser($id) {
		if ($this->User->delete($id)){
			$this->Session->setFlash('El usuario with id: '.$id.' fue borrado.');
			$this->redirect(array('action'=>'index'));
		}
	}
	
	function lista_profe(){
		$this->set('users', $this->User->find('all'));
	}

	function lista_videos(){
		$this->set('videos', $this->Video->find('all'));
	}
	
	function viewuser($id = null) {
		$this->User->id = $id;
		$this->set('user', $this->User->read());
	}
}
?>
\end{verbatim}

ProfeController.php: es este código se contiene las funciones para ver su perfil en viewuser() y transmitir
video a través de estreaming.\\
\footnotesize
\begin{verbatim}
<?php
class ProfeController extends AppController {
	var $helpers = array('Form', 'Html', 'Time');
	var $uses = array('Profe', 'Video', 'User','Carrera','Post','Alumno');
	public function isAuthorized($user){
		if(in_array($this->action, array('index','add','delete',
					'view', 'viewuser', 'viewprofe', 'transmitir')))
		{
			if($user['tipo'] == 0 )
			{
				return true;
			}
		}
		return false;
	}

	function index() {
	$this->set('videos', $this->Video->find('all',
				array('conditions' => array(
				'Video.user_id' => $this->Auth->User('id')))));
	$this->set('posts', $this->Post->find('all'));
	}

	function transmitir() {
	}

	function view($id = null) {
		$this->Video->id = $id;
		$this->set('video', $this->Video->read());
	}

	function viewuser() {
		$this->User->id = $this->Auth->User('id');
		$this->set('user', $this->User->read());	
	}
}
?>
\end{verbatim}

AlumnoController.php: en este código se maneja las acciones que realiza el ayudante, su perfil, 
la lista de profes, la lista de videos y la vista para ver el perfil del profe. \\
\footnotesize
\begin{verbatim}
<?php
class AlumnoController extends AppController {
	var $helpers = array('Form', 'Html', 'Time');
	var $uses = array('Alumno', 'Profe', 'Video', 'User', 
	'Historiale', 'Carrera', 'Departamento');
	public function isAuthorized($user)
	{
		if(in_array($this->action, array('index', 'lista_videos', 'view_video',
			'listacar', 'viewuser', 'viewprofe','lista_profe'))) 
		{
			if($user['tipo'] == 1 )
			{
				return true;
			}
		}
		return false;
	}

	function index() 
	{
		$this->set('profes', $this->User->find('all',array(
						'conditions' => array('User.tipo' => 0))));
		$this->set('carreras', $this->Carrera->find('all'));
		$this->set('departamentos', $this->Departamento->find('all'));
	}

	function lista_videos($id = null)
	{
		$this->User->id = $id;
		$this->set('videos', $this->Video->find('all', 
				array('conditions' => array('Video.user_id' => $id))));	
	}

	function lista_profe($id = null)
	{
		$this->User->id = $id;
		$this->set('profes', $this->User->find('all',
				array('conditions' => array('User.tipo' => 0))));	
	}

	function listacar($id = null){
	$this->set('videos', $this->Video->find('all', 
				array('conditions' => array('User.carrera_id' => $id))));	
	}

	function view_video($id = null) {
		$this->Video->id = $id;
		$this->set('video', $this->Video->read());
		$this->Historiale->saveField('user_id', $this->Session->read('Auth.User.id'));
		$this->Historiale->saveField('video_id', $id);
	}

	function viewuser() {
		$this->User->id = $this->Auth->User('id');
		$this->set('user', $this->User->read());
	}

	function viewprofe($id = null)
	{
	  $this->User->id = $id;
      $this->set('user', $this->User->read());
	}
}
?>
\end{verbatim}

VideoController.php: este código contiene todas las funciones relacionas con la administración del video, 
subir, borrar y ver el video.\\
\footnotesize
\begin{verbatim}
<?php
class VideoController extends AppController {
	var $helpers = array('Form', 'Html', 'Time','Session');
	var $name = 'Video';
	var $components = array ('Auth','Session');
	var $uses = array('User', 'Video','Alumno','Profe');

	function index() {
		$this->set('videos', $this->Video->find('all'));
	}

	function view_video($id = null) 
	{
		$this->Video->id = $id;
		$this->set('video', $this->Video->read());	
	}
																				
	function delete($id) 
	{
		if ($this->Video->delete($id))
		{
			$this->Session->setFlash('El video fue borrado.');
			$this->redirect(array('action'=>'index'));
		}
	}

	function add() 
	{
		if (!empty($this->data))
		{
			$currentFile = $this->params['data']['File'];
			$filePath = WWW_ROOT . 'videos/' . $currentFile['name']; 
			$movie = new ffmpeg_movie($currentFile['tmp_name']);
			$dur = $movie->getDuration();
			$hor = floor($dur/3600);
			$min = (($dur / 60) % 60);
			$sec = ($dur % 60);
			$duracion['hor'] = str_pad($hor, 2,"0", STR_PAD_LEFT);
			$duracion['min'] = str_pad($min, 2,"0", STR_PAD_LEFT);
			$duracion['sec'] = str_pad($sec, 2,"0", STR_PAD_LEFT);
			$duracion = implode(':',$duracion);
			 if (is_uploaded_file($currentFile['tmp_name']) && 
				 move_uploaded_file($currentFile['tmp_name'], $filePath) &&
 				 $this->Video->save($this->data) && 
				 $this->Video->saveField('duracion', $duracion) &&
 				 $this->Video->saveField('link', 'videos/' . $currentFile['name']))
 			{ 	
				$this->Session->setFlash('El video fue subido correctamente');
			} 
		}
	}
}
?>
\end{verbatim}

PostController.php: este código contiene todas las funciones relacionas con la administración del post, 
subir, borrar y ver el post.\\
\footnotesize
\begin{verbatim}
<?php class PostsController extends AppController {
	public $helpers = array('Html','Form','Session'); 
	var $components = array ('Auth','Session'); 
	var $uses = array('Post','User', 'Alumno');
	function index() {
		$this->set('posts', $this->Post->find('all'));
	}

	public function view($id = null) {
		$this->Post->id = $id;
	    $this->set('post', $this->Post->read());
	}

	public function add() {
		$this->Post->create();
		if ($this->request->is('post')) {
			if ($this->Post->save($this->request->data)) {
				$this->Session->setFlash('Tu post fue guardado');
				$this->redirect(array('action' => 'index'));
			}
		}
	}

	function edit($id = null) {
		$this->Post->id = $id;
	    if ($this->request->is('get')) {
			$this->request->data = $this->Post->read();
		} 
		else 
		{
			if ($this->Post->save($this->request->data)) 
				$this->Session->setFlash('Your post has been updated.');
				$this->redirect(array('action' => 'index'));
		}
	 }
	}

	function delete($id) {
	    if (!$this->request->is('post')) 
		{
			throw new MethodNotAllowedException();
		}
		if ($this->Post->delete($id)) 
		{
			$this->Session->setFlash('Post con id: ' . $id . ' ha sido borrado.');
			$this->redirect(array('action' => 'index'));
	  }
	}

	function lista_posts ($id = null){
		$this->User->id = $id;
		$this->set('posts', $this->Post->find('all', 
				array('conditions' => array('Post.user_id' => $id))));
	}
}
?>
\end{verbatim}

\normalsize
Las vistas son lo que los usuarios finales ven, acá se despliega la información obtenida de los controladores 
y se obtiene los datos para ser guardados en la Base de Datos, la interface para subir video y ver streaming. 
Los index son la página principal de cada modelo donde se comienza la navegación.\\

A continuación se describirán los códigos más importantes.\\

\paragraph{Del Modelo User:\\}

Login.ctp: este código contiene el formulario de ingreso a la pagina, los tres tipos de usuarios ingresan, 
a través de está intreface y son dirigidos hacia sus modelos correspondientes.\\

Add.ctp: este código despliega el formulario de ingreso de un nuevo usuario.\\

\paragraph{Del Modelo Alumno:\\}

Index.ctp: contiene la interface con las opciones del alumno, revisar la lista de profesores, ver 
su perfil o revisar su historial.\\

Los códigos de listas, despliegan la totalidad de registros asociados a la petición.\\

\paragraph{Del Modelo Profesor.\\}

Indes.ctp: este código contiene el despliegue de opciones para el profesor, como ver su perfil, video,
publicaciones y transmitir.\\

Transmitir.ctp: este código muestra la reproducción del video que se transmite y muestra el chat para que se 
pueda responder preguntas.\\

Los demás códigos realizan las acciones especificadas en el controlador.\\

\paragraph{Del Modelo Video:\\ }

View\_video: muetra la reproducción del video.\\

Add.ctp: contiene el código para subir un nuevo video.\\

\paragraph{Del modelo Post.\\}

Contiene las vistas para mostrar un publicación, subirla, editarla y borrarla.\\

\paragraph{Del servicio Websocket para el chat:\\}

Server.php: este código muestra las configuaraciones y funciones necesarias para el paso de mensajes 
generando un chat.\\

class.PHPWebSocket.php: este código contiene las funciones para trabajar con websocktes.\\

Fancywebsocket.js: contiene los script necesarios para el chat.\\

\paragraph{Del estilo:\\}

cake.generic.css: este código contiene la interfaz grafica para todas las paginas realizadas a través de CakePHP.\\

Los demás códigos contenidos en la carpeta de Cake, son propios del framework y ayudan a la funcionalidad de este, 
por ejemplo el Modelo Auth y su controlador, manejan el servicio de Autenticación, sin necesidad de programarlo.\\

\subsection{Otras Consideraciones}
-	Para completar el proyecto aun falta todo lo que son las vistas e interfaces de usuario, que es lo que quedaría 
para hacer en esta segunda parte del proyecto.\\

-	Sería ideal poder establecer esta aplicación para dispositivos móviles, en donde por ejemplo el 
cliente-profesor pudiera realizar una clase on-line a través de su móvil, ya sea un iphone, o ipad, smartphones, entre otros.\\

-	Los videos para ser cargados en el servidor deben ser en formato .mp4.\\

\newpage
\section{Conclusiones\\}
Respecto a lo que va del proyecto se puede indicar lo siguiente:\\
-	El proyecto está funcional, sus partes principales que son diferenciación de cuentas 
mediante login, acceso a las las distintas funciones pertinentes, poder subir y ver videos, realizar
el streaming-media, realizar publicaciones, entre otros, están operativas.\\

-	En general se puede decir que en cuanto a avance del proyecto este se encuentra en un 85\%, 
considerando que el resto del proyecto es generar las vistas de forma aun mas atractivas para el usuario,
el cual en este caso debieremos estar ayudados por algún dise\~nador gráfico o alguien entendido en el aspecto
visual.\\

-	En cuanto al desarrollo del proyecto, la parte más problemática fue la integración de las partes que
se encontraban por separado ya que, al utilizar el framework CakePHP, este tenia ciertas restricciones 
que nos limitaban, no era tan simple realizar las acciones en php simplemente y despuescomenzar a enlazar,
habia que entender a cabalidad primero como funcionaba el controlador del framework y la relacion junto con el 
modelo para luego generar las vistas.\\

-	Tambien a modo de crítica grupal es lo referente a la asignación de tareas, mejorar las habilidades blandas,
principalmente la comunicacion entre los integrantes del grupo, ya que hubieron varias descoordinaciones.\\

-	Y a modo de mejorar el sistema sería principalmente implementar de mejor forma el asunto de las vistas ya que 
aunque intentamos varios cambios en cuanto a mejorar el aspecto gráfico, no era del todo convincente las 
modificaciones que se realizaban en las plantillas css.\\

-	Repositorio Git en https://github.com/adansio/v-stream-o



\newpage
\bibliographystyle{abbrv}
%\bibliography{main}
\newpage
\begin{figure}
  \centering
      \includegraphics[width=1.0\textwidth]{Nav}
	    \caption{Esquema Navegacional.}
	\label{fig:Nav}
\end{figure}

\begin{figure}
  \centering
      \includegraphics[width=0.8\textwidth]{vistas/administrador}
	    \caption{Vista de Administrador.}
	\label{fig:Admin}
\end{figure}

\begin{figure}
  \centering
      \includegraphics[width=0.8\textwidth]{vistas/login}
	    \caption{Vista de login.}
	\label{fig:login}
\end{figure}

\begin{figure}
  \centering
      \includegraphics[width=0.8\textwidth]{vistas/listaprofesAdmin}
	    \caption{Lista de profesores desde Administrador}
	\label{fig:lista_prof_admin}
\end{figure}

\begin{figure}
  \centering
      \includegraphics[width=0.8\textwidth]{vistas/ListaPublicaciones}
	    \caption{Lista de Publicaciones}
	\label{fig:lista_pub}
\end{figure}

\begin{figure}
  \centering
      \includegraphics[width=0.8\textwidth]{vistas/ListaVideo}
	    \caption{Lista de Video}
	\label{fig:lista_vid}
\end{figure}

\begin{figure}
  \centering
      \includegraphics[width=0.8\textwidth]{vistas/Subir_Post}
	    \caption{Vista de Agregar un post como profesor.}
	\label{fig:add_post}
\end{figure}

\begin{figure}
  \centering
      \includegraphics[width=0.8\textwidth]{vistas/VerPost}
	    \caption{Vista de Post.}
	\label{fig:ver_post}
\end{figure}

\begin{figure}
  \centering
      \includegraphics[width=0.8\textwidth]{vistas/streamin}
	    \caption{Vista de Streaming.}
	\label{fig:streaming}
\end{figure}

\begin{figure}
  \centering
      \includegraphics[width=0.8\textwidth]{vistas/vervideo}
	    \caption{Vista de ver Video.}
	\label{fig:ver_video}
\end{figure}


\begin{figure}
  \centering
      \includegraphics[width=0.8\textwidth]{vistas/subir_video}
	    \caption{Vista de subir video como profesor.}
	\label{fig:add_video}
\end{figure}


\begin{figure}
  \centering
      \includegraphics[width=1.0\textwidth]{Clases}
	    \caption{Diagrama de Clases.}
	\label{fig:Clases}
\end{figure}

\begin{figure}
  \centering
      \includegraphics[width=0.6\textwidth]{secuencia/AlumnoPublicaciones}
	    \caption{Secuencia de Alumno Publicaciones.}
	\label{fig:al_publ}
\end{figure}

\begin{figure}
  \centering
      \includegraphics[width=0.6\textwidth]{secuencia/AlumnoVideo}
	    \caption{Secuencia Alumno Video.}
	\label{fig:al_vid}
\end{figure}

\begin{figure}
  \centering
      \includegraphics[width=0.6\textwidth]{secuencia/BorrarUsuario}
	    \caption{Secuencia Borrar Usuario}
	\label{fig:sec_del_user}
\end{figure}

\begin{figure}
  \centering
      \includegraphics[width=0.6\textwidth]{secuencia/BorrarVideo}
	    \caption{Secuencia de Borrar Video.}
	\label{fig:del_vid}
\end{figure}

\begin{figure}
  \centering
      \includegraphics[width=0.6\textwidth]{secuencia/IngresarUsuario}
	    \caption{Secuencia Ingresar Usuario.}
	\label{fig:ing_user}
\end{figure}
\clearpage
\begin{figure}
  \centering
      \includegraphics[width=0.6\textwidth]{secuencia/PublicacionesProfe}
	    \caption{Secuencia Publicaciones Profesor.}
	\label{fig:sec_pub_prof}
\end{figure}

\begin{figure}
  \centering
      \includegraphics[width=0.6\textwidth]{secuencia/SubirVideoProfe}
	    \caption{Secuencia de Subir Videos.}
	\label{fig:sub_vid}
\end{figure}

\begin{figure}
	\centering
      \includegraphics[width=0.6\textwidth]{secuencia/TransmitirProfe}
	    \caption{Secuencia de Streaming desde profesor.}
	\label{fig:str_pr}
\end{figure}

\begin{figure}
  \centering
      \includegraphics[width=0.7\textwidth]{vistas/modelodedatos}
	    \caption{Modelo de datos.}
	\label{fig:moddatos}
\end{figure}


\end{document}

