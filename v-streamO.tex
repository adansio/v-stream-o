\documentclass[12pt]{article}
\usepackage[T1]{fontenc}
\usepackage[utf8]{inputenc}
\usepackage[spanish]{babel}

\title{V-stream'o}
\author{Luis González - José Jorquera - Adan Morales}
\date{\today}

\begin{document}
\maketitle
\thispagestyle{empty}

\newpage
\section{Introducción}
\subsection{Trasfondo}
Existen momentos durante el periodo universitario que por uno u otro motivo no es posible asistir a clases 
``claves'' para la comprensión a cabalidad de los contenidos de una asignatura, ya sea por ejemplo un 
compromiso de fuerza mayor, o alguna eventualidad como sucedió el a\~no 2011 con el movimiento estudiantil, 
el cual en varios casos no fue posible entregar completamente o explicar como corresponde alguna materia 
en particular, o quizas algún contenido es frecuentemente consultado, y el profesor desearía tener una 
explicación mas a fondo de algun tema y compartirla con sus alumnos pero bajo ciertas condiciones de uso
y de reservación de derechos.\\
Es por esto que se plantea un sistema en el cual mediante videos (grabaciones o streming media) el profesor
podrá compartir ``clases'' a sus alumnos, tal como existen ejemplos de profesores de Massachusetts Institute of Technology (MIT) que publican sus clases en sitios como www.youtube.com ayudando a comprensión de temas
no solo a usuarios del MIT si no que a nivel mundial.

\subsection{Resumen}
El proyecto consta de un sistema de acceso web, en el cual un profesor (registrado en el sistema) pueda 
publicar videos de ciertas asignaturas o algún contenido en específico a modo de apoyo para la comprensión
de un tema en particular, y además realizar video conferencias programadas para sus alumnos. Junto con esto
el alumnado (matriculado en un curso con un respectivo profesor) podrá acceder a estos contenidos mediante
el sistema, este será de un ambiente similar a lo que es el sitio www.aula.usm.cl, agregando esta nueva
funcionalidad de video, agregando a esto cualquier nueva duda proveniente del video, se dispondrá del mail 
del profesor para realizarle consultas, y en el caso de que el recurso sea streaming, se contará con un campo
especial de mensajeria para relizarle consultas directamente durante la clase en vivo a traves del sistema
V-stream'O.

\newpage
\section{Descripción General}

\subsection{Objetivos}
\subsubsection{Generales}
Como se ha mencionado anteriormente el objetivo general del sistema es acercar el conocimiento de los 
profesores a sistemas actuales de comunicación para sus alumno, como es la publicación de video o de streming
mediante el sistema interno V-stream'O.
\subsubsection{Específicos}
\begin{enumerate}
\item Crear un portal web en el cual se puedan publicar videos asociados a asignaturas específicas o 
complementarias de la UTFSM.
\item Alumnos registrados en el sistema puedan tener acceso a los videos especificados según las
restricciones pertinentes.
\item Establecer sistema de video conferencias en la cual el profesor pueda mediante una webcam local
realizar una ``clase on-line''.
\item Los alumnos puedan observar las ``clases on-line'' y realizar consultas al profesor a través de un 
sistema de chat interno.
\end{enumerate}

\subsubsection{Problemática que enfrenta}
El Problema a enfrentar en este proyecto es el de poder ayudar en la formación académica de un alumno de la 
UTFSM, debido a veces en que por alguna razón externa a la vida académica, o quizas para los
deportistas que deben ir a representar a la universidad en campeonatos, o dentro de los distintos talleres 
que la universidad dispone, algunos alumnos no podrán
asistir a alguna clase de real importancia dejándolo posiblemente sin algún conocimiento elemental, aun
cuando es responsabilidad de este consultar sobre los temas no comprendidos.

Tambien se puede dar el caso que el profesor no alcance a cubrir todo un tema con la profundidad que desea
dentro del tiempo estipulado pedagógicamente, por lo cual quedan ciertos temas como "en el aire", o
simplemente no se cubren.


\subsection{Descripción de la solución}
A modo de solución a la problematica anterior, se establece este sistema, de forma que (ayudados de la buena
voluntad de los profesores en cuanto a la contribución de material ya sea fundamental o complementario) 
eventualmente en algún momento el profesor publique material 
audiovisual en un servidor establecido para dicho propósito, y así dicho alumno pueda acceder al sistema 
y recibir de mejor forma dicho conocimiento, o mejor aún, el profesor ayudado mediante el sistema V-stream'O podría
acordar y luego realizar "una clase o ayudantia on-line".

\subsection{Tecnologías usadas}
- Servidor\\
- Lenguaje en servidor php\\
- Modelamiento de base de datos sqlDesigner\\
- Manejo de base de datos MySQL\\
- Framework de desarrollo Cakephp\\ 


\newpage
\section{Especificacón de los requerimientos}
\subsection{Usuarios del sistema}
Alumnos

Los usuarios mayoritariamente beneficiados con este sistema son los alumnos, ya que principalmente
el proyecto está enfocado a ellos, en el cual podrán obtener los recursos multimedia ya sea video
o una clase via streaming, además de poder realizar consultas mientras se desarrolla dicha clase.

Profesores

Los usuarios que aportaran en que el sistema cumpla el objetivo general, son lo profesores, 
encargados de publicar periodicamente videos de interés, ya sean de orden importante o complementarios, 
ademas de realizar "clases on-line" cuando ellos estimen conveniente, o corresponda por alguna situción
en particular.
\subsection{Descripción de los requerimientos}
\subsubsection{Funcionales}
Permitir al usuario no registrado acceder a cierto contenido público.\\
Permitir al usuario registrado matricularse en un curso.\\
Permitir al usuario registrado buscar videos en los cuales tenga acceso.\\
Permitir al usuario registrado ver videos.\\
Permitir al usuario registrado mantener un historial con los utimos videos observados.\\ 
Permitir al usuario registrado, modificar su portada de contenidos.\\

\subsubsection{No Funcionales}
De Buen rendimiento.
Seguro.
Accesible.
Operativo.
Escalable.
Administrable.
\subsection{Tareas de Usuario}
\subsection{Funciones del Sistema}

\newpage
\section{Dise\~no de la interfaz de usuario}
\subsection{Esquema Navegacional}
\subsection{Prototipos de Pantallas}

\newpage
\section{Dise\~no del Sistema}
\subsection{Dise\~no de la Arquitectura}
\subsection{Dise\~no Lógico}
\subsubsection{Diagrama de Clases}
\subsubsection{Modelo de Datos}
\subsubsection{Diagramas de Flujo}

\newpage
\section{Implementación}
\subsection{Descripción de las Componentes}
\subsection{Otras Consideraciones}

\section{Conclusions}\label{conclusions}


\newpage
\bibliographystyle{abbrv}
%\bibliography{main}

\end{document}

